\documentclass{bioinfo}
\copyrightyear{2012}
\pubyear{2012}

\begin{document}
\firstpage{1}

\title[Pathway-based visualization of cross-platform microarray datasets]{Pathway-based visualization of cross-platform microarray datasets}
\author[Clemens Wrzodek \textit{et~al}]{Clemens Wrzodek\,$^{1\footnote{to whom correspondence should be addressed}}$, Johannes Eichner\,$^{1}$ and Andreas Zell\,$^1$}

\address{$^{1}$Center for Bioinformatics Tuebingen (ZBIT), \\University of Tuebingen, 72076 T\"ubingen, Germany}
\history{Received on XXXXX; revised on XXXXX; accepted on XXXXX}

\editor{Associate Editor: XXXXXXX}

\maketitle

\begin{abstract}

\section{Motivation:}
Text Text Text  Text Text Text Text Text Text Text Text
Text  Text Text Text Text Text Text Text Text Text  Text Text Text Text Text Text Text Text Text  Text Text Text Text Text Text Text Text Text  Text Text Text Text Text Text Text Text Text  Text Text Text Text Text Text Text Text Text  Text Text Text Text Text.

\section{Results:}
Text  Text Text Text Text Text Text Text Text Text  Text Text Text Text Text Text Text Text Text  Text Text Text Text Text Text Text Text Text  Text Text Text Text Text Text

%\section{Availability:}
%The described method is included in the InCroMAP application, which is available at \href{TODO}{TODO}.

\section{Contact:} \href{clemens.wrzodek@uni-tuebingen.de}{clemens.wrzodek@uni-tuebingen.de}
\end{abstract}

\section{Introduction}

The first generation of microarray platforms was developed as a high-throughput technique for
profiling the transcriptome of diverse biological systems (i.e., cells, organs or organisms) under
various experimental conditions (TODO: Refs. Golub, etc.). As these traditional gene-centered arrays
were mostly limited to mRNA transcripts, the vast majority of visualization tools are still focused
on mRNA datasets (Refs.). To date, a plethora of different microarray platforms are readily
available. These include gene-centered platforms which rely on current genome annotations as well as
unbiased tiling arrays which interrogate large non-repetitive regions of the genome. Diverse types
of platforms have been specificly designed for the interrogation of different genomic features,
ranging from mRNA or miRNA transcripts, through proteins or protein modifications, to relevant
functional elements such as exons, SNPs or promoters (TODO: Refs). In addition to arrays serving for
the quantification of global gene expression on the RNA or protein level, also epigenetic
modifications such as DNA methylation (DNAm) can be monitored on a genome-wide level using
microarray technology \citep{Hoheisel2006}.  

% Motivation für die Entwicklung des Tools 
Several tools exist for the visual inspection of datasets from individual platforms
(Refs.). However, the current inventory of publicly available tools, which are capable of
integrating and jointly visualizing data from multiple microarray platforms, is still very limited.
Here, we introduce a method, for integrated pathway-centered visualization of datasets generated
from the same biological samples using different microarray platforms, which interrogate
complementary genomic and epigenomic features.

% Pathway-basierte Visualisierung vs. Regions-basierte Visualisierung 
In contrast to commonly used region-based visualization methods (e.g., \citep[see][]{UCSCBrowser},
TODO: weitere Beispiele und Zitate), we propose to visualize the microarray data in the context of
specific signaling or metabolic pathways, which can in many cases be more easily related to the
biological problem under study than individual genes or genomic regions.  


% TODO: das h�chste der gef�hle sind methoden um 2 farben in 1 knoten, aber keiner kann DNA
% methylation oder sogar miRNA knoten rein.

% TODO: Johannes: Kannst du 1-2 S�tze jeweils zu Cytoscape, Ingenuity, eventuell noch Cytoscape +
% Plugins schreiben?

% TODO: Related Work. Abgrenzung zu GenMAPP und Ingenuity, Cytoscape, KEGG Atlas, KEGG Array; Siehe
% auch Kohlbacher-Nils Paper, S. Symons paper .Gibt es ueberhaupt ein Tool, welches alle 4
% Datentypen (miRNA, etc) visualisieren kann?

% TODO: InCroMAP vs. publizierte Pathway-basierte Visualisierungstools 
There are other tools, specialized in pathway analysis (e.g., Ingenuity, ), or in pathway
visualization (Cytoscape, KEGG Atlas). Some even offer visualizing data in a pathway (GenMAPP, KEGG
Array, Symons programm). No method today for high-dimensional, heterogeneous cross-platform
datasets.

% Integrated oder single platform enrichment zur Identifikation der relevanten Pathways The pathways
which are relevant for the conducted experiment can be deduced from the differentially expressed
genes using pathway enrichment analysis. For this purpose, candidate pathways which are putatively
involved in the studied biological phenomenon are typically ranked according to p-values resulting
from a hypergeometric test for overrepresentation. The results are usually presented to the user as
a sorted table or barplot which does not show any superordinate relations of the pathways detected
as enriched with differentially expressed genes. In addition to this traditional approach, we also
implemented an alternative method, which provides the user with a more structured view of the
metabolic pathways linked to a certain microarray experiment. Owing to the hierarchical structure of
the KEGG PATHWAY database, InCroMAP can visualize the enrichments computed for each individual
metabolic pathway in the context of the higher-order overall metabolic pathway map compiled by KEGG
(http://www.genome.jp/kegg/pathway/map/map01100.html). Starting from either a ranked pathway table
or a colored meta-pathway map, individual pathways can be visualized in InCroMAP and overlaid with
microarray data from multiple platforms, to facilitate thorough visual inspection of the measured
pathway alterations. In contrast to previous work (TODO: Refs zu Cytoscape, Explain, etc.), InCroMAP
offers convenient functions to overlay a pathway plot with sample-matched microarray data from
platforms measuring gene regulation on mRNA, miRNA, DNAm and protein level.


%\begin{methods} 
\section{Methods} 

% Import von Microarray-Daten in InCroMAP 
Before use with InCroMAP the microarray datasets of interest have to be preprocessed and annotated
using platform-specific workflows (Refs zu Ringo, MEDME, AgiMicroRna, Limma, Annotate). These
workflows usually involve (1) the quality control of the raw data (TODO: cite arrayQualityMetrics,
simpleaffy, etc.), (2) the data normalization to correct for background noise and experimental
variation (TODO: cite RMA and GCRMA paper or review paper about normalization), and (3) the mapping
of the probes to genes or genomic regions. After these preprocessing steps the microarray data has
to be exported in tabular format (e.g., CSV or Excel). These tables have to contain two types of
columns: (1) annotation columns, containing probe or probeset IDs (e.g., Affymetrix IDs) and
database IDs of the corresponding genes (e.g., Ensembl or Entrez IDs), and (2) data columns
containing either fold-changes and/or p-values resulting from basic statistical analysis of the
microarray data. For convenience, these tables can be imported into InCroMAP per drag-and-drop, and
the tool tries to auto-detect the platform type.  

% Import von Pathways in InCroMAP 
The pathway data is automatically imported from KEGG. In the KEGG PATHWAY database each pathway map
is available for download as a KGML document which is internally converted into a GraphML document
by InCroMAP. These GraphML files can be interactively visualized by InCroMAP using the yWorks graph
viewer (TODO: Ref.). In order to overcome limitations of the KGML format, one can create an overlay
graph that shows the original KEGG pathway image in the background, which may provide the user with
additional information about cellular structure and compartmentalization.  

% Visualisierung von Microarray-Daten im Kontext von Pathways 
The nodes are then overlaid with microarray data such to reflect mRNA expression, protein expression
and DNA methylation changes. As a last step, nodes are being added for miRNAs and miRNA expression
data is visualized in the pathway. See Figure~\ref{fig:visualization_steps} for a graphical
description of all those visualization steps.

\subsection{Pathway visualization} 

The basic prerequisite for pathway-based visualization is visualizing the pathway itself. We are
using KEGGtranslator \citep[see][]{Wrzodek2011} to perform a basic conversion of the KEGG KGML
documents to GraphML and to annotate all nodes with entrez gene identifiers. In short,
KEGGtranslator converts all KGML entries to nodes and all relations to edges. Some basic errors are
corrected automatically and appropriate shapes, colors and labels are inferred. Then, all nodes are
annotated with various identifiers and further information. The resulting document is the base for
our visualizations. At this point, it is important to note that KEGG usually draws referenced
pathways as rectangular nodes with rounded corners, small molecules as circles, and single gene
products (e.g., enzymes) as well as gene families as rectangles. This means that one rectangular
node can consist of multiple different enzymes, depending on the KEGG definition.

# TODO: Auch Schritt von figure a) zu b) erwaehnen. Manche Informationen manchmal nur in grafischen
# abbildungen enthalten, deshalb original KEGG bild als hintergrund nehmen und den eigentlichen
# graphen als overlay schicht darueber anzeigen. Auf figure 1 a), b) fuer beispiel verweisen.

\subsection{Visualization of messenger RNA expression data} 

Visualization of mRNA datasets is straightforward by somehow gene-centering the input data and then
changing the node color according to this value. As input, this method requires processed mRNA
datasets with gene identifiers and ``observations". In this context, ``observations" can be any
statistical significance or comparative measure (we used \emph{p}-values or fold-changes). Then, for
each node in the pathway, one value must be calculated. Therefore, all probes that belong to a node
are gathered from the input dataset and then the mean or median is calculated. Another possibility
is to take the most important probe (i.e., $\min\emph{p}-value$ or $\max|fold-change|$) to get a
single value for each pathway node.  This value is then used to calculate a color for the node. For
fold-changes (which are usually $log_2$ values), we color every node with a fold-change $\geq2$ red
and all fold-changes $\leq-2$ blue. Fold-changes of zero are defined to have a white color and
colors between $\pm2$ are faded from blue or red to white, depending on the actual fold-change. The
same procedure can be used for \emph{p}-values, except that just one minimum threshold and one
minimum color must be defined. Furthermore, the color for \emph{p}-values should not be changed on a
linear, but on a log-scale. See Figure~\ref{fig:visualization_steps}c) for an example of visualized
mRNA data.

\subsection{Visualization of protein and protein modification expression data}

Visualization of protein datasets is performed by adding small boxes below pathway nodes and
changing the color of the boxes according to the corresponding protein expression data. Protein
datasets usually have identifiers, like Entrez Gene IDs, UniProt IDs, etc. which allows to make a
straightforward mapping to pathway nodes. Then, all values must be collected and a color must be
calculated for each node in the same way as already described for mRNA datasets.

Protein modification datasets must be treated differently. They usually not only contain one
expression values for the basic form of the protein, but also for some phosphorylated or likewise
modified form. Therefore, separate boxes are created below each pathway node for all
modifications. These boxes are labeled according to the modification. Furthermore, the color for
each box must only be calculated on probes that match the pathway node and the modification of each
box.

\subsection{Visualization of DNA methylation data}

Ein Wert nur als Hinweis, hier geht etwas [click gibt details?]. fold-change wird zu box von -2 bis
+2, p-value im grunde ein bar-blot von 1 bis 0.00005 oder so...

Einzelner Wert mit binning und $\frac{\sum\limits_{i=1}^n\log_2 x}{n}$, f�r fold-changes oder so
peak detection m�glich und max. peak anzeigen.

\textbf{Johannes.}


\subsection{Visualization of micro RNA expression data}

Visualizing micro RNA (miRNA) datasets is not straightforward, because pathways usually do not
contain miRNAs. Pathway mainly consist of small molecules and enzymes, which are products of protein
coding genes. Therefore, to add miRNAs to a pathway, a connection must be established from miRNAs to
to protein coding genes.

Biologically, miRNAs are small non-coding RNAs that regulate gene expression by binding to mRNA
targets and somehow inducing a degradation of the targeted mRNA \citep{Bartel2004}. The targets for
each miRNA must be known and there are several databases that contain information about
experimentally verified miRNA targets (e.g., miRecords \citep[see][]{miRecords}, miRTarBase
\citep[see][]{miRTarBase}, TarBase \citep[see][]{TarBase}) or predicted miRNA targets
\citep{Alexiou2009}. We use these miRNA target databases to perform the linkage between miRNAs and
pathways.

Pathway-based visualization of miRNA datasets is done by annotating all known targets to every miRNA
in the input dataset. Then all miRNAs that have targets in any pathway that should be visualized are
added to the pathway as small triangular nodes. The relation to the pathway is then established by
creating an edge from every miRNA to every target in the pathway. The triangular miRNA nodes
themselves are colored according to their expression, as described for mRNA.  This leads to an
integrated visualization that contains miRNA expression, miRNA target relation information and (if
also mRNA data is visualized) the expression of the targeted
mRNA. Figure~\ref{fig:visualization_steps}f) shows an example result of the described procedure.


\subsection{OFFENE FRAGEN} 

Sollten wir hier einfaerbung nach enrichment p-values bzw. den "metabolic
pathways"-pathway erwaehnen? Oder lieber fuer spaetere publikationen "aufspaaren"?

%\end{methods}



\section{Results and discussion}

# TODO: Gesamtkonzept und Ergebnisse / Bilder vorstellen
# hier erw�hnen, dass methoden in InCroMAP drin sind? oder lieber in conslutions?

\begin{figure*}[t] \centering \includegraphics[width=1.0\textwidth]{figures/visualization-steps.png}
\caption{ TODO: Bild zeigt AUSSCHNITT des MAPK signaling pathway. Alle Schritte von a) bis f) kurz
erlauetern und kurz sagen, dass f) quasi dem finalen Bild dieser Methode
entspricht. }\label{fig:visualization_steps} \end{figure*}


# TODO: - Sollen wie noch "Table 1" aus MARCAR D4.1 reinnehmen (tabellarische erklaerung der
# visualisierung aller 4 datentypen mit beispielen), und/oder - Ein schoenes Gesamtbild e.g. von cell
# cycle oder wnt signaling oder sowas [ - den metabolic pathway overview mit eingefarbten knoten] Dies
# nur wenn diese Methode auch noch bestandteil dieses papers wird (muesste noch bei methoden
# usw. aufgenommen und erklaert werden).


\section{Conclusion}

# TODO: Zusammenfassung, vor allem Einzigartigkeit der Methode (insb. miRNA, (DNAm)) herausstellen,
# und kurze interpretation eines finalen bildes liefern ("man sieht in einem bild pathway
# informationen, miRNA relationen, expressionsdaten von miRNA, mRNA, bla; und sonst ist meist nach dem
# enrichment schluss bzw mRNA einfarbung bereits das maximum moegliche").



\section*{Acknowledgement} 
We gratefully acknowledge contributions from Andreas Dr\"ager and Finja B\"uchel, as well as the
whole MACRCAR consortium.

\paragraph{Funding\textcolon} 
The research leading to these results has received funding from the Innovative Medicine Initiative
Joint Undertaking (IMI JU) under grant agreement nr. 115001 (MARCAR project).

\bibliographystyle{natbib}
%\bibliographystyle{achemnat}
%\bibliographystyle{plainnat}
%\bibliographystyle{abbrv}
%\bibliographystyle{bioinformatics}
%\bibliographystyle{plain}

\bibliography{document}
%TODO: When paper is ready, comment \bibliography line and add content of document.bbl here.

\end{document}
